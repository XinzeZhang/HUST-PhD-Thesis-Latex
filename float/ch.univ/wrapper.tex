
% \begin{figure}
%     % \centering
%     \begin{minipage}[t]{0.45\textwidth}
%         \raggedleft
%         \begin{subfigure}[t]{\linewidth}
%             \begin{python}
%                 from task.parser import get_parser
%                 from task.TaskWrapper import Task
%                 if __name__ == "__main__":
                    
%                     args = get_parser()
                
%                     args.cuda = True
%                     args.datafolder = 'paper.sm'
%                     args.dataset = 'mg'
%                     args.H = 84
%                     args.model = 'esn'
%                     args.rep_times = 20
%                     args.metrics = ['rmse','nrmse', 'mape']
                    
%                     task = Task(args)
%                     task.tuning()
%                     task.conduct()
%                     task.evaluation(args.metrics)
                
%                 \end{python}
%                 \caption{}
%         \end{subfigure}
        
%     \end{minipage}
%     \hfill
%     \begin{minipage}[t]{0.45\textwidth}
%         \begin{subfigure}[t]{\linewidth}
%             \begin{python}
%                 from task.parser import get_parser
%                 from task.TaskWrapper import Task
%                 if __name__ == "__main__":
                    
%                     args = get_parser()
            
%                     task = Task(args)
%                     task.tuning()
%                     task.conduct()
%                     task.evaluation(args.metrics)
                
%                 \end{python}
%           \caption{b}
%         \end{subfigure}
%         \\
%         \begin{subfigure}[b]{\linewidth}
%             \begin{bash}
%                 python task.py -datafolder paper.sm -dataset mg -H 84 -model esn -rep_times 20 -metrics rmse nrmse mape
%             \end{bash}
%             \caption{c}
%         \end{subfigure}
%     \end{minipage}

%     \caption{MG数据集上构造ESN预测模型进行提前84步预测建模任务}
%   \end{figure}

\begin{figure}[t!]
    \begin{minipage}[b]{0.45\textwidth}
        \begin{python}
            from task.parser import get_parser
            from task.TaskWrapper import Task

            if __name__ == "__main__":
                
                args = get_parser()
        
                task = Task(args)
                task.tuning()
                task.conduct()
                task.evaluation(args.metrics)
            
            \end{python}
        \caption{预测建模框架的主文件:task.py \label{fig:ch.univ.task}}
    \end{minipage}
    \hfill
    \begin{minipage}[b]{0.45\textwidth}
        \begin{python}
            from task.TaskLoader import Opt
            import importlib
            import ...

            class Task(Opt):
                def __init__(self, args):
                    
                    self.data_config(args)
                    self.model_config(args)
                    self.exp_config(args)

                ...
            \end{python}
        \caption{TaskWrapper中的Task类 \label{fig:ch.univ.init}}
    \end{minipage}
\end{figure}

\begin{figure}[t!]
    \begin{python}
        from task.parser import get_parser
        from task.TaskWrapper import Task
        if __name__ == "__main__":
            
            args = get_parser()
        
            args.cuda = True
            args.datafolder = 'paper.sm'
            args.dataset = 'mg'
            args.H = 17
            args.model = 'esn'
            args.rep_times = 20
            args.metrics = ['rmse','nrmse', 'mape']
            
            task = Task(args)
            task.tuning()
            task.conduct()
            task.evaluation(args.metrics)
        
        \end{python}
        \caption{自定义主文件task.py传入任务参数的示例\label{fig:ch.univ.mg} }
    \end{figure}

    \begin{figure}[t!]
        \begin{bash}
            python task.py -datafolder paper.sm -dataset mg -H 17 -model esn -rep_times 20 -metrics rmse nrmse mape
        \end{bash}
            \caption{由命令行传入任务参数的示例\label{fig:ch.univ.bash} }
        \end{figure}