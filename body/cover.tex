
%%% Local Variables:
%%% mode: latex
%%% TeX-master: t
%%% End:

\ctitle{基于随机映射的时间序列深度学习预测建模技术}

\xuehao{D201880***} 
% \xuehao{D201880970} 
\schoolcode{10487}
\csubjectname{管理科学与工程} 
\cauthorname{张心泽}
\csupervisoronename{{鲍玉昆}}
\miji{\hei\textbf{公开}}
% \cauthorname{***}
% \csupervisorname{***} 
\csupervisoronetitle{教 授}
\csupervisortwoname{{李 四}}
\csupervisortwotitle{副教授}
\defencedate{2023~年~4~月~26~日} \grantdate{}
\chair{}%
\firstreviewer{} \secondreviewer{} \thirdreviewer{}


\etitle{Stochastic Deep Learning Based Time Series
Forecasting Techniques}
\edegree{Doctor of Philosophy in
    Management}
\esubject{Management Science and Engineering}
\eauthor{Zhang Xinze}
\esupervisorone{Prof. Bao Yukun}
\esupervisortwo{Assoc. Prof. Li Si}
% \eauthor{***** *****}
% \esupervisor{Prof. *** *****}
\edate{April, 2023}

\dctab{\begin{tabular}{|P{0.9cm}|P{1.8cm}|P{1.8cm}|P{5.4cm}|}
    \hline
    &{\hei\textbf{姓名}}&{ \hei\textbf{职称}}&{\hei \textbf{单位}}\\
    \hline
    主席&X\hspace{1em}X&教授&武汉大学~信息管理学院\\
    \hline
    \multirow{4}{*}{委员}&X\hspace{1em}X&教授&武汉大学~信息管理学院\\
    \cline{2-4}
    &XXX&教授&华中科技大学~管理学院\\
    \cline{2-4}
    &XXX&教授&华中科技大学~管理学院\\
    \cline{2-4}
    &X\hspace{1em}X&教授&华中科技大学~管理学院\\
    \hline
\end{tabular}
}


%定义中英文摘要和关键字
\cabstract{
    时间序列预测建模技术是能源、金融和公共卫生等众多应用领域的重要技术。针对时间序列预测建模问题,深度学习预测建模技术因其优异的预测性能成为了近年的研究热点。然而,以卷积神经网络和循环神经网络为代表的深度神经网络预测模型受限于梯度下降权重训练方法,需要在反复训练权重参数的基础上才可具备优异的预测性能,因此带来昂贵的计算消耗;加之深度学习预测模型复杂的结构参数设置,使得深度学习预测建模技术面临建模效率低与模型选择难等问题。随机映射方法作为神经网络模型的一种非迭代训练方法,通过固定模型随机初始化的输入层与隐藏层权重,利用简单直接的闭式求解算法计算模型输出层权重,以此完成权重参数训练过程,因而可有效提升神经网络模型的构造效率。

    本学位论文基于随机映射方法对时间序列深度学习预测建模技术展开研究,针对随机深度神经网络预测模型在卷积结构和循环结构上的模型选择特有个性问题,提出随机卷积隐藏结构的构造优化方法和随机循环输出结构的构造优化方法;在此基础上探索随机深度神经网络特定代表结构的共性优化问题,提出随机深度神经网络预测模型一般结构下的二重特征选择方法和混合结构下的生长优化方法,进一步提升模型预测性能;在技术方法研究的同时,结合流感阳率预测、原油价格预测、电力负荷预测与电力价格预测等重要应用场景进行应用研究。本文的主要研究工作和创新性成果如下:


    首先,针对卷积结构的随机深度神经网络预测模型构造与优化问题,提出基于误差反馈随机建模和贪心优化的新颖构造与优化方法,解决了既有方法的参数选择不足及输出权重病态矩阵难题。通过向隐藏层中递归添加随机卷积核,利用最小二乘法局部更新输出权重,构造出随机卷积隐藏结构,并证明了该构造方法的收敛性;借助贪心搜索优化卷积核参数,使其在单卷积层内自适应选择不同宽度的卷积核,具备了不同尺度局部特征的学习能力。在合成数据和多项现实数据上的实验表明,该模型具有极高的模型构造与选择效率,表现出与梯度下降训练的深度学习预测模型媲美甚至更优的预测性能。

    其次,针对循环结构的随机深度神经网络预测模型构造与优化问题,提出了基于状态遮掩和粒子群优化的新颖构造与优化方法,解决了既有方法忽视的输出结构优化问题。基于对循环神经网络输出结构的随机映射实现,分析得出不同输出结构的异同优劣;通过向循环隐藏特征表示施加状态掩码,提出状态遮掩的随机循环输出结构,借助粒子群优化自适应选择循环隐藏特征表示和目标的映射取舍,提升了输出结构的学习能力。在人工合成、电力负荷和室外温度数据集上的实验表明,与既有随机循环神经网络及其输出结构优化方法相比,基于所提方法构造的模型具备明显的多步预测优势。

    再次,针对一般结构的随机深度神经网络预测模型特征选择问题,提出了基于二重特征结构选择和树状结构帕森斯估计的新颖特征选择方法,消除了既有选择方法的一维输入特征结构局限。结合深度神经网络的时步多维度读取能力,利用滑动窗口构造时步多维度的二维输入特征结构;通过向每一输入时步添加时步掩码,建立基于时步维度与时步掩码组合的二重特征结构表示,借助树状结构帕森斯估计自适应优化特征结构参数,降低了抽象特征的学习难度。在人工合成、电力负荷和电力价格数据集上的实验表明,与既有卷积与循环结构随机深度神经网络及其特征选择方法相比,基于所提方法优化的模型具有更好的预测准确性与稳定性。

    最后,针对混合结构的随机深度神经网络预测模型构造与优化问题,提出了基于误差反馈生长和三阶段优化的新颖混合结构构造与优化方法,消除了既有方法的单一结构局限。通过递归添加不同结构的随机子网络,利用岭回归局部更新输出权重,构造出鲁棒的混合隐藏结构,并证明了该构造方法的收敛性;通过预优化、子训练和调正则的三阶段优化方法对迭代添加的子网络进行完整优化,使其自适应选择出多种结构的子网络及参数,融合了不同深度结构的学习能力。在人工合成、空气污染和电力负荷数据集上的实验表明,与多种既有随机深度神经网络模型相比,基于所提方法构造的模型具有更好的预测性能。
}

\ckeywords{时间序列预测;深度学习;随机映射;卷积神经网络;回声状态网络}

\eabstract{
    Time series forecasting is of great importance for a learning system in dynamic environments, playing a vital role in many real-world applications, such as energy, traffic, finance, and industry. Recent studies have shown that deep learning technique has shown intriguing prediction performance, leading to extensive research on the applications of the DNN models for time series forecasting. However, represented by the convolutional neural network and recurrent neural network models, the deep neural network--based forecasting models have complex architecture-related parameters and rely on gradient-based algorithms to train the weight-related parameters, making it extremely time-consuming and challenging to well establish a forecasting model. As an alternative method of training the neural network model, the stochastic mechanism fixes the weights of the input layer and the hidden layer after random initialization, and uses a simple and direct closed-form solution algorithm to calculate the weight of the output layer of the model, which can effectively improve the construction efficiency of the neural network model. 
    
    Therefore, this study investigates time series deep learning prediction modeling technology based on the stochastic mechanism, pays attention to the structure selection problems under the specific structures, and proposes the hidden structure construction method of the stochastic convolutional neural network and the output structure selection method of the stochastic recurrent neural network. On this basis, the optimization problems under the universal structure are explored, and the feature selection method as well as the parameter optimization method of the stochastic deep neural network prediction model are proposed to further improve the prediction performance of the models. At the same time, the application research is carried out in combination with important application scenarios, such as influenza prediction, crude oil price prediction, electricity load prediction, electricity price prediction, and so on. The main contributions of this dissertation are summarized as follows:
    
    Focusing on the model construction problem of stochastic convolutional neural network--based forecasting model, a novel error-feedback stochastic modeling strategy and greedy-based selection algorithm are proposed to craft the random convolutional neural network for time series forecasting. The proposed method suggests that random filters and neurons of the error-feedback fully connected layer are incrementally added to steadily compensate for the prediction error during the construction process, and then a greedy-based filter selection is introduced to enable the model to extract the different sizes of temporal features. Comprehensive experiments on the simulated dataset and several real-world datasets show that the proposed method exhibits stronger predictive power and lower computing overhead compared to trained state-of-the-art deep neural network models.

    Focusing on the model construction problem of stochastic recurrent neural network--based forecasting model, a novel state mask strategy with particle swarm optimization is proposed to construct the random recurrent output structure for time series forecasting. Based on the investigation of the stochastic implementation of different recurrent output structures of training-based recurrent neural networks, the proposed method adds a mask to each step of the recurrent hidden features, and then a particle swarm optimization based mask selection is introduced to evolve the mapping relationships from recurrent hidden features to their targets, which improves the learning ability of the stochastic recurrent output architecture. Compared with the stochastic recurrent neural networks with the existing output architecture selection method, comprehensive experiments on the simulated dataset, electricity load dataset, and outside temperature dataset demonstrate the superiority of the proposed method. 

    Focusing on the input feature selection problem of stochastic deep neural network--based forecasting model, a novel dual feature-structured selection method with tree-structured parzen estimator is proposed. Based on the ability of deep neural architecture that can model multiple dimensions in each input time step, a multiple-step-dimension two-dimensional feature structure is established with a moving window schema. The proposed method adds a mask to each input step to represent the two-dimensional feature structure with the combination of step dimension and step mask, and then tree-structured parzen estimator is introduced to evolve the feature structure, which improves the learning ability of the stochastic deep neural networks. Compared with the stochastic deep neural networks with the existing feature selection method, comprehensive experiments on the simulated dataset, electricity load dataset, and electricity price dataset demonstrate the superiority of the proposed method.

    Focusing on the model construction problem of stochastic deep neural network--based forecasting model, a novel error-feedback triple-phase optimization strategy is proposed to grow stochastic deep neural network--based predictor with mixed deep neural architectures. The proposed method incrementally adds diverse deep subnetworks to the network, where the output weights of the subnetworks are calculated via ridge regression to improve the robustness of the constructed model, and the parameters of the subnetworks are evolved with pre-tuning, sub-tuning, and reg-tuning optimization, making the network take advantage of different deep neural architectures. Compared with the existing stochastic deep neural networks, comprehensive experiments on the simulated dataset, air pollution datasets and electricity load datasets demonstrate the superiority of the proposed method.
}

\ekeywords{
    Time series forecasting; deep learning; stochastic mechanism; convolutional neural network; echo state network
}
