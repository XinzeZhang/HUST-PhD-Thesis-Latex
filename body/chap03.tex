\chapter{实验研究类论文*}
\label{cha:thirdsection}


\section{引言(引言标题可选)}
\label{sec:method}
由于绪论中已对全文相关的研究背景和进展做了综述,因此,每章的引言中,请用1页左右的版面写一个导引,简要说明本章研究的背景或动机,以起到承上启下的作用,不宜太长。

引言的最后一段,说明本章的主要内容,如拟基于什么理论或方法,针对什么问题开展研究。注意:这里不能给结论。

若少于1个页面时,建议省略标题“引言”,直接在章的标题下写上几段话即可。

若学位论文属实验研究类论文,且论文所用的实验材料、仪器设备、实验方法基本一样,但应用不同,此时可考虑在第二章用整章的篇幅来描述,这样后面的章节就无需描重复描述相同内容,以避免冗余。

若实验性研究论文所用的材料与方法并非完全相同时,建议在各章中分别介绍材料与方法、实验结果、分析与讨论,最后是小结。



\section{材料与方法}
\label{sec:algorithm}

\subsection{实验材料}
针对材料的描述,一般应给出材料来源。

\subsection{实验仪器}
针对仪器的描述,若为商用仪器,给出型号、品牌、产地。

若为自行搭建的系统,则需要简要说明介绍组成及原理。

\subsection{实验方法}
介绍本章所涉及的实验方法或实验步骤。

\subsection{数据处理(根据情况确实是否需要)}
主要介绍对所获得的数据是如何处理的。

\section{结果1(请拟定具体的题目)}
作为章节的二级标题,若直接采用“结果与分析”,则在目录中看不到有效信息。为避免出现毫无辨识度的标题,建议将所得的结果作为二级标题,但也不要一幅图一节,将相关的结果整合为2-3节即可,各节篇幅长短不宜悬殊太大。

\section{结果2(请拟定具体的题目)}

\section{结果3(请拟定具体的题目)}

\section{分析与讨论}
论文排版时,无论因为图表等原因还是其它原因,除章与章的之间存在分页,空白处的地方不要太多。

讨论部分的关键在于:分析本章所得到的一些结果之间的关联性;所得到的结果与文献结果是否一致?分析产生一致或不一致结果的原因,以此体现论文的创新点。也可以指出本章中的不足,并分析原因及未来改进措施。

\section{本章小结}
本章主要介绍实验研究类的论文正文章节的框架结构。在每章的最后,都需要对该章的内容进行小结,不宜太长,建议1/2-2/3页版面较好。主要小结一下本章用什么理论或方法、做了什么事、得到的重要结果或结论。
