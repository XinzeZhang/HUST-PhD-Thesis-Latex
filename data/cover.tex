
%%% Local Variables:
%%% mode: latex
%%% TeX-master: t
%%% End:

\ctitle{华中科技大学博士学位论文~\LaTeX{} 模板使用示例文档}

\xuehao{D2008XXXX} \schoolcode{10487}
\csubjectname{计算机系统结构} \cauthorname{黄齐丘}
\csupervisorname{黄厚德} \csupervisortitle{教授}
\defencedate{2013~年~1~月~27~日} \grantdate{}
\chair{}%
\firstreviewer{} \secondreviewer{} \thirdreviewer{}

\etitle{A Sample Document of \LaTeX{} Template for Doctoral Thesis of
Huazhong University of Science and Technology} \edegree{Doctor of Philosophy in
Engineering} \esubject{Computer Architecture} \eauthor{Huang Qiqiu} \esupervisor{Prof. Huang Houde}


%定义中英文摘要和关键字
\cabstract{
本文基于清华大学学位论文~\LaTeX+CJK
模板(薛瑞尼版本)和华中科技大学博士学位论文~\LaTeX+CJK 模板(1.0
版本,姜峰), 主要用来展示华中科技大学博士学位论文~\LaTeX+CJK
模板(2.0 版本), 并简要介绍其使用方法。

使用该模板生成的论文已获得华中科技大学研究生院的认可,符合相应的博士学位论文
的格式。这篇文档按照博士学位论文的要求生成,具体使用方法请参看本文源文件。

一般而言,中文摘要包含500-1000字,1-2页。关键词5-10个。}

\ckeywords{\LaTeX~\keyspace CJK~\keyspace 华中科技大学~\keyspace
博士学位论文~\keyspace 模板}

\eabstract{ The purpose of this document to present and summarize
the \LaTeX Template for Doctoral Thesis of Huazhong University of
Science and Technology, which is mainly based on Thesis Template of
Tsinghua University (Xue Ruini's version) and Doctoral Thesis Template
of HUST (1.0 version, Jiang Feng).

A thesis produced using this template has been approved by the
Graduate School of Huazhong University of Science and Technology,
and fulfils the corresponding formatting requirements. This document
is generated according to the format of Doctoral Thesis. Please
refer to the source file for usage guidelines.

Generally, the abstract and the key words should be consistent with
the Chinese version.}

\ekeywords{\LaTeX\keyspace CJK\keyspace HUST\keyspace Doctoral
Thesis\keyspace Template}
